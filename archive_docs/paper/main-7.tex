% !TEX program = xelatex
\documentclass[aspectratio=1610,8pt]{beamer}
\usepackage[utf8]{inputenc}
\usepackage[T1]{fontenc}
\usepackage{lmodern}
\usepackage{caption,subcaption}
\usepackage{multirow}
\usepackage{multicol}
\usepackage{comment}
\usepackage{csquotes}
\usepackage[maxbibnames=99]{biblatex}
\usepackage[makeroom]{cancel}
\usepackage{threeparttable}
\usepackage{tikzsymbols}
\usepackage{textcomp}
\usepackage{parskip}
\usepackage{pgf}
\usepackage{color,soul}
\usepackage[listings,theorems]{tcolorbox}
\tcbuselibrary{skins}
\usepackage{hyperref}
\usepackage{xcolor,soul}
\usepackage{empheq}
\usepackage{enumerate}
\usepackage{collcell}
\usepackage{booktabs}
\usepackage{etoolbox}
\usepackage{xcoffins}
\usepackage{pgfpages}
\usepackage{stackengine,tikz}
\usepackage{transparent}
\usepackage{eqnarray,amsmath}
\usepackage{amsfonts}
\usepackage{amssymb}
\usepackage{mathtools}
\usepackage{expl3}
\usepackage[vietnamese]{babel}

\usetheme{Madrid}

% Modern color scheme
\definecolor{primaryblue}{RGB}{0,90,156}
\definecolor{accentblue}{RGB}{100,143,255}
\definecolor{textgray}{RGB}{66,66,66}
\definecolor{lightgray}{RGB}{245,245,245}
\definecolor{cvut_navy}{HTML}{0065BD}
\definecolor{cvut_blue}{HTML}{6AADE4}
\definecolor{lightgreen}{HTML}{90EE90}
\definecolor{darksalmon}{rgb}{0.91, 0.59, 0.48}
\definecolor{darkred}{RGB}{180,0,0}
\definecolor{darkgreen}{RGB}{0,120,0}

% Block colors
\AtBeginDocument{
  \setbeamercolor{block title}{use=structure,fg=white,bg=primaryblue}
  \setbeamercolor{block body}{use=structure,fg=black,bg=white}
  \setbeamercolor{block title alerted}{use=alerted text,fg=white,bg=red!75}
  \setbeamercolor{block body alerted}{fg=black,bg=white}
  \setbeamercolor{block title example}{use=example text,fg=white,bg=accentblue}
  \setbeamercolor{block body example}{fg=black,bg=lightgray}
}

% Other color configurations
\sethlcolor{lightblue}
\setbeamercolor{section in toc}{fg=black,bg=yellow} 
\setbeamercolor{alerted text}{fg=cvut_blue}
\setbeamercolor{palette primary}{bg=primaryblue,fg=white}
\setbeamercolor{palette secondary}{bg=primaryblue,fg=white}
\setbeamercolor{palette tertiary}{parent=palette primary}
\setbeamercolor{palette quaternary}{fg=primaryblue,bg=lightgray}
\setbeamercolor{sidebar}{fg=primaryblue,bg=white}
\setbeamercolor{titlelike}{parent=palette primary}
\setbeamercolor{frametitle}{bg=white,fg=primaryblue}
\setbeamercolor{B}{bg=accentblue!20,fg=textgray}
\setbeamercolor{itemize item}{fg=primaryblue}

\hypersetup{
    urlcolor=blue, 
    linkcolor=darksalmon, 
    citecolor=blue,
    colorlinks=true
}

\addbibresource{references.bib}

\renewcommand<>{\hl}[1]{\only#2{\beameroriginal{\hl}}{#1}}
\newcommand{\boxedeq}[2]{\begin{empheq}[box={\fboxsep=6pt\fbox}]{align}\label{#1}#2\end{empheq}}

\setbeamertemplate{caption}[numbered]
\useoutertheme{infolines}
\setbeamertemplate{page number in head/foot}[framenumber]
\setbeamertemplate{section in toc}[default]
\setbeamertemplate{itemize item}[circle]
\setbeamertemplate{subsection in toc}[subsections numbered]
\setbeamertemplate{navigation symbols}{} 
\setbeamertemplate{caption}{\raggedright\insertcaption\par}
\usefonttheme{professionalfonts}

\setbeamertemplate{headline}{%
\begin{beamercolorbox}[colsep=1.5pt]{upper separation line head}
\end{beamercolorbox}
\begin{beamercolorbox}{section in head/foot}
    \vskip2pt\insertsectionnavigationhorizontal{\paperwidth}{}{\hskip0pt plus1filll}\vskip2pt
\end{beamercolorbox}%
\begin{beamercolorbox}[colsep=1.5pt]{lower separation line head}
\end{beamercolorbox}
}

\setbeamercovered{dynamic}

%====================================================
%========== TITLE INFORMATION =======================
%====================================================
\institute[HCMUT]{
\vspace{-1cm}
\begin{center}
    \includegraphics[height=1.25cm]{Beamer/hcmut.png}
\end{center}
\vspace{0.1cm}
\Large{\scriptsize{\textbf{\color{black}BÁO CÁO THỰC TẬP 1}}} \\ \textbf{BÁO CÁO LẦN 2}\\
% \textbf{Visual Grounding for Hallucination Reduction}
\\[0.5cm]
\scriptsize{
\begin{table}[h]
\centering
    \begin{tabular}{ll}
        \color{blue}Học viên thực hiện: & \textbf{Võ Phạm Tuấn Dũng - 2570015} \\
    \end{tabular}
\end{table}
}
}

\title[]{\normalsize\textsf{TRƯỜNG ĐẠI HỌC BÁCH KHOA - ĐHQG TP.HCM\\KHOA KHOA HỌC VÀ KỸ THUẬT MÁY TÍNH}}

\date[Oct 2025]{ \\ \scriptsize{TP.HCM, tháng 1 năm 2025}}

%====================================================
\AtBeginSection[]{
  \begin{frame}<beamer>
    \frametitle{Nội dung}
    \tableofcontents[currentsection,currentsubsection]
  \end{frame}
}

\begin{document}

\begin{frame}[plain]{}
	\titlepage
\end{frame}

\begin{frame}[plain]{\textbf{Nội dung báo cáo}}
\tableofcontents[]
\end{frame}





%====================================================
\section{Review 2 papers}

% \subsection{CoRGI: Modular Verification}

\begin{frame}{CoRGI: Kiến trúc 3 stage}
\textbf{Paper}: Yi \& Shang (2025) \parencite{yi2025corgi} - "Verified Chain-of-Thought Reasoning with Visual Grounding"

\begin{block}{Core Innovation}
Three-stage modular pipeline, explicit verification cho từng visual evidence ứng với từng bước suy luận.
\end{block}

\vspace{0.5em}
\begin{columns}[T]
\column{0.48\textwidth}
\textbf{Stage 1: Reasoning Chain Generation}

VLM mạnh (Qwen-2.5VL-7B) sinh ra chuỗi reasoning:
$$R = \{r_1, r_2, ..., r_n\}$$

Ví dụ với câu hỏi "What is in the scene?":
\begin{itemize}
    \item $r_1$: "There is a person"
    \item $r_2$: "The person is holding an object"
    \item $r_3$: "The object appears to be a cup"
\end{itemize}

Lưu ý: ở stage này chưa có grounding, chỉ là textual reasoning.

\begin{figure}
        \centering
        \includegraphics[width=0.48\linewidth]{Screenshot 2025-10-08 at 16.38.59.png}
        \caption{Minh hoạ 3 stage}
        \label{fig:placeholder}
\end{figure}


\column{0.48\textwidth}
\textbf{Stage 2: Visual Evidence Verification Module (VEVM)}

\textbf{2.1 Relevance Classification}
\begin{itemize}
    \item Aim: Deciding if and how much
to look
    \item Lightweight MLP classifier (tự train)
    \item Binary gate: step này cần visual verification không?
    \item Đo mức độ quan trọng của evidence (importance score)
    % \item Importance score: "75\%" - mức độ quan trọng
    % \item 
\end{itemize}

\textbf{2.2 RoI Selection (Hybrid):}
\begin{itemize}
    \item Aim: Deciding where to look
    \item Nếu có explicit reference (e.g., "person 0") → dùng GT boxes
    \item Nếu implicit → dùng Grounding DINO \cite{ren2024groundingdino15} zero-shot
    \item Pragmatic: tận dụng annotations khi có
\end{itemize}

\textbf{2.3 VLM Fact-Checking:}
\begin{itemize}
    \item Aim: Describing
what is seen
    \item Pre-trained VLM mô tả RoI content
    \item Objective verification
\end{itemize}



\end{columns}
\end{frame}

\begin{frame}{CoRGI: Stage 3 và Experimental Results}


\begin{columns}[T]
\column{0.5\textwidth}



\textbf{Stage 3: Final Answer Synthesis}

Aggregates thông tin từ:
\begin{itemize}
    \item Question $Q$
    \item Reasoning chain $R = \{r_1, ..., r_n\}$
    \item Visual Evidence $E = \{e_1, ..., e_n\}$ (mỗi $e_i$ có importance score)
\end{itemize}

VLM tổng hợp bộ thông tin trên để tạo ra final answer kèm lời giải thích.
% , trong đó visual evidence được weighted theo importance scores.

\vspace{0.5em}
% \textbf{Ví dụ synthesis:}
% \begin{itemize}
%     \item $r_2$ (75\%): "person holding object"
%     \item $e_2$: "image shows person with hands near cup"
%     \item Verification: consistent ✓
% \end{itemize}

\begin{figure}
    \centering
    \includegraphics[width=0.5\linewidth]{Screenshot 2025-10-08 at 17.00.34.png}
    \caption{3 module con của stage 2}
    \label{fig:placeholder}
\end{figure}
        

\column{0.5\textwidth}
\textbf{Results trên VCR benchmark} \parencite{yi2025corgi}:

\begin{table}
\tiny
\begin{tabular}{@{}lc@{}}
\toprule
\textbf{Method} & \textbf{Q→A Accuracy} \\
\midrule
Qwen-2.5VL Raw & 61.0\% \\
+ Standard CoT & 61.3\% (+0.3\%) \\
\textbf{+ CoRGI} & \textbf{63.3\%} (\textcolor{darkgreen}{+2.0\%}) \\
\midrule
Gain over baseline & \textcolor{darkgreen}{+2.3 pts} \\
Gain over CoT & \textcolor{darkgreen}{+2.0 pts} \\
\bottomrule
\end{tabular}
\end{table}

% \textbf{Human evaluation} (100 samples, 5 raters) \parencite{yi2025corgi}:
% \begin{itemize}
%     \item Factuality: \textcolor{darkgreen}{4.52/5}
%     \item Helpfulness: \textcolor{darkgreen}{4.18/5}
%     \item Inter-rater agreement: moderate
% \end{itemize}

\vspace{0.5em}
\textbf{Ablation studies Q->A} \parencite{yi2025corgi}:
\begin{itemize}
    \item Remove relevance classifier → -1.9\%
    \item Remove RoI selection → -1.3\%
    \item Remove VLM fact-checking → -2.2\%
\end{itemize}
\end{columns}
\end{frame}

\begin{frame}{Minh hoạ inference}
\begin{figure}
    \centering
    \includegraphics[width=0.35\linewidth]{Screenshot 2025-10-08 at 17.18.11.png}
    \caption{Minh hoạ inference trên tập VQA-v2, ảnh lấy từ paper}
    \label{Minh hoạ inference trên tập VQA-v2, ảnh lấy từ paper}
\end{figure}
\end{frame}

\begin{frame}{CoRGI: Critical Assessment}
\begin{columns}[T]
\column{0.48\textwidth}
\textbf{Strengths (Điểm mạnh):}

\begin{enumerate}
    \item \textbf{Modular design}: Có thể swap components dễ dàng. Ví dụ thay Grounding DINO 1.5 \cite{ren2024groundingdino15} bằng Florence-2\cite{xiao2024florence2} mà không cần retrain.
    
    \item \textbf{Training-free core}: Chỉ train relevance classifier nhỏ, các thành phần chính dùng pre-trained models.

    \item \textbf{Transparent reasoning}: User nhìn thấy từng step: reasoning → relevance score → RoI → evidence → answer.
    
    \item \textbf{Generalizes}: Test trên cả VCR và VQA-v2, cho thấy không overfit benchmark.
\end{enumerate}

\column{0.48\textwidth}
\textbf{Weaknesses (Điểm yếu):}

\begin{enumerate}
    \item \textcolor{darkred}{\textbf{Post-hoc verification}}: Reasoning chain đã được generate trước, nếu sai từ đầu thì verification không cứu được.
    
    \item \textcolor{darkred}{\textbf{Sequential bottleneck}}: Phải chạy VLM nhiều lần (Stage 1, 2.3, 3) → computational cost cao.
    
    \item \textcolor{darkred}{\textbf{VCR-specific advantages}}: Dataset VCR có GT boxes cho explicit references, advantage này không có ở general scenarios.
    
    \item \textcolor{darkred}{\textbf{Error propagation}}: Nếu Grounding DINO detect sai object, evidence sẽ misleading.
\end{enumerate}
\end{columns}

% \vspace{0.5em}
% \begin{alertblock}{Key insight từ ablations}
% All three VEVM components necessary - không component nào redundant \parencite{yi2025corgi}. Điều này chứng minh design rationale solid.
% \end{alertblock}
\end{frame}


% \subsection{MM-GCoT: Consistency Paradigm}

\begin{frame}{MM-GCoT: Training paradigm mới}
\textbf{Paper}: Wu et al. (2025) \parencite{wu2025gcot} - "Grounded Chain-of-Thought for Multimodal Large Language Models"

\begin{block}{Core Innovation}
Đề xuất training paradigm cho step-wise grounded reasoning với explicit bbox supervision. Quan trọng hơn, giới thiệu \textcolor{primaryblue}{consistency metric} - thước đo critical nhưng bị bỏ qua trước đây.
\end{block}

\vspace{0.5em}
\begin{columns}[T]
\column{0.5\textwidth}
\textbf{Formulation transformation:}

\textbf{Naive approach:}
$$y = f(I, Q)$$
Model nhận image $I$ và question $Q$, output trực tiếp answer $y$.

\textbf{GCoT approach:}
$$y = f_n(...f_2(f_1(I, Q, V_1), V_2)..., V_n)$$

Trong đó $V_i$ là visual evidence (bounding boxes) tại step $i$. Multi-step process với grounding tại mỗi bước.

\column{0.5\textwidth}
\textbf{Dataset construction} \parencite{wu2025gcot}:

\begin{itemize}
    \item \textbf{Scale}: 24,022 examples, 5,033 unique images
    \item \textbf{Source}: Visual Genome dataset
    \item \textbf{3 categories}:
    \begin{itemize}
        \item Attribute: "What color is the car at [x,y]?"
        \item Judgment: "Is the description 'red car' correct?"
        \item Object: "What object is at location [x,y]?"
    \end{itemize}
    \item \textbf{Annotation}: 4-stage pipeline với IoU-based region-description matching
\end{itemize}

\textbf{Training objective:}
$$P(y, \{bbox_i\} | I, Q)$$
Model học jointly generate answer và bounding boxes.
\end{columns}
\end{frame}

\begin{frame}{MM-GCoT: The Shocking Paradox}
\begin{columns}[T]
\column{0.5\textwidth}
% \textbf{Baseline performance} (trước training) \parencite{wu2025gcot}:

% \begin{table}
% \scriptsize
% \begin{tabular}{@{}lcc@{}}
% \toprule
% \textbf{Model} & \textbf{Answer} & \textbf{Consistency} \\
%  & \textbf{Accuracy} & \\
% \midrule
% LLaVA-7B & 66.4\% & 29.3\% \\
% LLaVA-13B & 68.3\% & 27.8\% \\
% Qwen2.5-VL-7B & 75.7\% & \textcolor{darkgreen}{67.3\%} \\
% \textcolor{darkred}{Qwen2.5-VL-72B} & \textcolor{darkgreen}{81.7\%} & \textcolor{darkred}{24.6\%} \\
% \midrule
% InternVL2.5-78B & 75.7\% & 35.4\% \\
% \bottomrule
% \end{tabular}
% \end{table}

% \begin{alertblock}{The Paradox}
% \textbf{72B model}: Accuracy cao nhất (81.7\%)\\
% NHƯNG: Consistency thấp nhất (24.6\%)\\
% → \textcolor{darkred}{18.2\% worse than 7B model!}
% \end{alertblock}
\begin{alertblock}{The Paradox}
\textbf{Phát hiện quan trọng từ nghiên cứu gần đây}:

Thí nghiệm cho thấy các mô hình hiện đại đạt answer accuracy từ 75-82 \% phần trăm trên visual question answering tasks, nhưng answer-grounding consistency chỉ đạt 24-48 \% \parencite{wu2025gcot}.
\newline
Điều này chứng tỏ các mô hình đang "đoán đúng" câu trả lời mà không thực sự dựa vào đúng visual evidence, một dạng hallucination tinh vi mà các accuracy metrics truyền thống không phát hiện được.
\end{alertblock}

\column{0.5\textwidth}
% \textbf{Interpretation} \parencite{wu2025gcot}:

% Larger models có thể:
% \begin{itemize}
%     \item \textbf{Overfit} to training data distribution
%     \item Rely on \textbf{statistical patterns} thay vì visual grounding
%     \item "Answer correctly for wrong reasons"
%     \item Higher capacity → easier to memorize shortcuts
% \end{itemize}

% \vspace{0.5em}
% \textbf{Sau GCoT training} \parencite{wu2025gcot}:

% \begin{table}
% \scriptsize
% \begin{tabular}{@{}lcc@{}}
% \toprule
% \textbf{Model} & \textbf{Answer} & \textbf{Consistency} \\
% \midrule
% LLaVA-7B GCoT & 85.4\% & 85.0\% \\
% LLaVA-13B GCoT & 86.8\% & 89.0\% \\
% \midrule
% Gain & +18-19\% & \textcolor{darkgreen}{+55-61\%} \\
% \bottomrule
% \end{tabular}
% \end{table}

% Consistency improvement gấp đôi accuracy improvement!
\end{columns}
\end{frame}



%====================================================
%========== SECTION: BENCHMARKS & DATA ==============
%====================================================
\section{Hệ thống Benchmarks}





%====================================================
%========== DATA COLLECTION STRATEGY ================
%====================================================
% \section{Kế hoạch Thu thập và Xử lý Dữ liệu}

% \begin{frame}{Data Collection Strategy: Tổng quan tiếp cận}
% \begin{block}{Philosophy và constraints}
% Trong phạm vi học kỳ 8 tuần với limited resources (single GPU T4/L4), chúng ta không thể train models from scratch trên large-scale datasets. Strategy focus vào: (1) sử dụng pre-existing datasets effectively, (2) tạo small curated evaluation sets cho specific failure modes, và (3) leverage pre-trained models với minimal fine-tuning khi cần.
% \end{block}

% \vspace{0.3em}
% \begin{columns}[T]
% \column{0.5\textwidth}
% \textbf{Three-tier Data Strategy}:

% \begin{enumerate}
%     \item \textbf{Tier 1 - Primary Datasets} (Sử dụng trực tiếp):
%     \begin{itemize}
%         \item RefCOCO+ test set: 5,726 samples
%         \item POPE: 3,000 samples (500 images × 6 questions)
%         \item MM-GCoT test set: 994 samples
%         \item \textcolor{darkgreen}{No preprocessing needed}, publicly available
%     \end{itemize}
    
%     \item \textbf{Tier 2 - Subset Sampling} (Thu gom subsets):
%     \begin{itemize}
%         \item GQA validation: Sample 5,000 from 134K
%         \item VCR validation: Sample 3,000 from 26K
%         \item ScienceQA: Sample 2,000 from 4,241 test
%         \item \textcolor{accentblue}{Stratified sampling} by difficulty và question type
%     \end{itemize}
    
%     \item \textbf{Tier 3 - Targeted Creation} (Tạo mới nếu cần):
%     \begin{itemize}
%         \item Failure cases từ Tier 1 evaluation
%         \item 100-200 samples cho specific error patterns
%         \item Manual annotation với bounding boxes
%     \end{itemize}
% \end{enumerate}

% \column{0.48\textwidth}
% \textbf{Data Sources và Accessibility}:

% \begin{table}
% \tiny
% \begin{tabular}{@{}lcc@{}}
% \toprule
% \textbf{Dataset} & \textbf{License} & \textbf{Size (Test)} \\
% \midrule
% RefCOCO+ & MIT & 5.7K \\
% POPE & MIT & 3K \\
% MM-GCoT & Academic & 994 \\
% GQA & CC BY 4.0 & 134K (val) \\
% VCR & CC BY 4.0 & 26K (val) \\
% Visual Genome & CC BY 4.0 & N/A (images) \\
% ScienceQA & Apache 2.0 & 4.2K \\
% HallusionBench & MIT & 1.1K \\
% \bottomrule
% \end{tabular}
% \end{table}

% \textbf{Download Sources}:
% \begin{itemize}
%     \item Hugging Face Datasets hub
%     \item Official GitHub repositories
%     \item Papers with code links
% \end{itemize}

% \textbf{Storage Requirements}:
% \begin{itemize}
%     \item Images: ~20GB (deduplicated)
%     \item Annotations: ~500MB
%     \item Total: <25GB feasible
% \end{itemize}
% \end{columns}
% \end{frame}

% \begin{frame}{Data Processing Pipeline}
% \begin{columns}[T]
% \column{0.5\textwidth}
% \textbf{Stage 1: Download và Verification} (Week 1-2):

% \begin{enumerate}
%     \item \textbf{Script 1 - Batch Download}:
%     \begin{itemize}
%         \item Sử dụng Hugging Face \texttt{datasets} library
%         \item Parallel download với checkpointing
%         \item MD5 checksum verification
%     \end{itemize}
    
%     \item \textbf{Script 2 - Data Validation}:
%     \begin{itemize}
%         \item Verify image integrity (PIL loading test)
%         \item Check annotation format consistency
%         \item Count samples per category
%         \item Flag missing or corrupted files
%     \end{itemize}
    
%     \item \textbf{Script 3 - Deduplication}:
%     \begin{itemize}
%         \item Identify shared images across datasets
%         \item Create symlinks thay vì duplicate storage
%         \item Build unified image index
%     \end{itemize}
% \end{enumerate}

% \textbf{Expected Output}: Verified dataset catalog với statistics report.

% \column{0.48\textwidth}
% \textbf{Stage 2: Preprocessing} (Week 2-3):

% \begin{enumerate}
%     \item \textbf{Image Preprocessing}:
%     \begin{itemize}
%         \item Resize to max dimension 1024px (preserve aspect ratio)
%         \item Convert to RGB nếu cần
%         \item Save standardized format (JPEG quality 95)
%     \end{itemize}
    
%     \item \textbf{Annotation Standardization}:
%     \begin{itemize}
%         \item Convert all bbox formats về [x\_min, y\_min, x\_max, y\_max]
%         \item Normalize coordinates về [0, 1] range
%         \item Create unified JSON schema
%     \end{itemize}
    
%     \item \textbf{Subset Sampling Strategy}:
%     \begin{itemize}
%         \item GQA: Stratified by question type (compositional, logical, relation)
%         \item VCR: Balanced Q→A, QA→R difficulty levels
%         \item Ensure coverage of all question categories
%     \end{itemize}
% \end{enumerate}

% \textbf{Tools}: Python scripts với libraries: \texttt{PIL}, \texttt{pandas}, \texttt{json}, \texttt{tqdm}.

% \textbf{Quality Control}: Manual spot-check 100 random samples per dataset.
% \end{columns}
% \end{frame}

% \begin{frame}{Timeline và Milestones: 8-Week Plan}
% \begin{table}
% \scriptsize
% \begin{tabular}{@{}p{1.5cm}p{3.5cm}p{4cm}p{2cm}@{}}
% \toprule
% \textbf{Week} & \textbf{Tasks} & \textbf{Deliverables} & \textbf{Checkpoint} \\
% \midrule
% \textbf{W1-2} & \textbf{Data Preparation} \newline - Download Tier 1 datasets \newline - Verify integrity \newline - Setup storage structure \newline - Document statistics & - Dataset catalog \newline - Verification report \newline - Storage mapping & âœ" All datasets downloaded \newline âœ" <25GB total \\
% \midrule
% \textbf{W3} & \textbf{Preprocessing} \newline - Standardize formats \newline - Sample Tier 2 subsets \newline - Create dataloaders & - Processed datasets \newline - Sampling scripts \newline - DataLoader code & âœ" Unified format \newline âœ" Dataloaders work \\
% \midrule
% \textbf{W4-5} & \textbf{Baseline Evaluation} \newline - Run models trên Tier 1 \newline - Collect metrics \newline - Identify failure modes & - Baseline results \newline - Error analysis \newline - Failure case catalog & âœ" All 5 core benchmarks \newline âœ" Consistency gap measured \\
% \midrule
% \textbf{W6} & \textbf{Pipeline Implementation} \newline - Integrate grounding \newline - Test trên small subset \newline - Debug issues & - Working pipeline \newline - Integration tests \newline - Preliminary results & âœ" Pipeline runs E2E \newline âœ" No crashes \\
% \midrule
% \textbf{W7} & \textbf{Full Evaluation} \newline - Run pipeline trên all benchmarks \newline - Ablation studies \newline - Tier 3 creation nếu cần & - Complete results \newline - Ablation analysis \newline - Targeted samples & âœ" w/ vs w/o grounding \newline âœ" Statistical significance \\
% \midrule
% \textbf{W8} & \textbf{Analysis và Documentation} \newline - Error taxonomy \newline - Comparative analysis \newline - Final report & - Research report \newline - Presentation slides \newline - Code release & âœ" Reproducible \newline âœ" Documented \\
% \bottomrule
% \end{tabular}
% \end{table}

% \vspace{0.3em}
% \textbf{Risk Mitigation}:
% \begin{itemize}
%     \item Buffer time: Mỗi task có 10-20\% time buffer cho unexpected issues
%     \item Fallback plan: Nếu Tier 2 sampling quá time-consuming, focus chỉ vào Tier 1 (vẫn sufficient cho 5 core benchmarks)
%     \item Parallel tasks: Data download (W1) có thể song song với literature review
% \end{itemize}
% \end{frame}

% \begin{frame}{Critical Considerations và Best Practices}
% \begin{columns}[T]
% \column{0.5\textwidth}
% \textbf{Data Quality Checks}:

% \begin{enumerate}
%     \item \textbf{Annotation Quality}:
%     \begin{itemize}
%         \item RefCOCO có 14-24\% annotation errors \parencite{yu2016modeling}
%         \item Manual verification cho ~100 samples
%         \item Use Ref-L4 cleaned version nếu available
%     \end{itemize}
    
%     \item \textbf{Train/Test Leakage Prevention}:
%     \begin{itemize}
%         \item Strictly use designated test splits
%         \item Document any overlap between datasets
%         \item Never tune trên test set
%     \end{itemize}
    
%     \item \textbf{Difficulty Distribution}:
%     \begin{itemize}
%         \item Sample across all difficulty levels
%         \item Ensure không bias về easy samples
%         \item Report per-difficulty breakdown
%     \end{itemize}
% \end{enumerate}

% \column{0.48\textwidth}
% \textbf{Reproducibility Guidelines}:

% \begin{enumerate}
%     \item \textbf{Documentation Requirements}:
%     \begin{itemize}
%         \item Dataset versions và download dates
%         \item Preprocessing steps chi tiết
%         \item Random seeds cho sampling
%         \item Hyperparameters cho mọi experiments
%     \end{itemize}
    
%     \item \textbf{Code Organization}:
%     \begin{itemize}
%         \item Separate scripts: download, preprocess, evaluate
%         \item Config files cho all settings
%         \item Requirements.txt with pinned versions
%     \end{itemize}
    
%     \item \textbf{Result Archiving}:
%     \begin{itemize}
%         \item Save raw predictions (JSON)
%         \item Log all metrics với timestamps
%         \item Archive model checkpoints
%     \end{itemize}
% \end{enumerate}

% \vspace{0.3em}
% \begin{alertblock}{Critical Reminder}
% Focus on \textcolor{primaryblue}{consistency metrics} từ đầu. Setup evaluation framework để measure answer-grounding consistency alongside standard accuracy \parencite{wu2025gcot}. Đây là key differentiator của grounded reasoning research.
% \end{alertblock}
% \end{columns}
% \end{frame}

% \begin{frame}{Summary: Benchmarks và Data Collection}
% \begin{block}{Key Takeaways}
% Comprehensive evaluation của grounded reasoning systems đòi hỏi systematic approach với multiple benchmarks covering reasoning, grounding, hallucination, và consistency dimensions. Data collection strategy phải balance giữa comprehensiveness và practical feasibility trong 8-week timeframe.
% \end{block}

% \vspace{0.3em}

% % \begin{columns}[T]
% % \column{0.48\textwidth}

% \textbf{Benchmark Strategy Summary}:

% \begin{itemize}
%     \item \textbf{Minimum suite}: 5 core benchmarks (MM-GCoT, POPE, GQA, RefCOCO+, VCR)
%     \item \textbf{Extended suite}: +3 benchmarks cho comprehensive evaluation
%     \item \textbf{Critical metrics}: Report answer accuracy, grounding accuracy, VÀ consistency
%     \item \textbf{Estimated time}: 2-3 days evaluation trên T4/L4 GPU
% \end{itemize}

% \textbf{Data Collection Summary}:

% \begin{itemize}
%     \item \textbf{Three-tier strategy}: Primary (direct use), Subset (sampling), Targeted (creation)
%     \item \textbf{Storage}: <25GB total requirement
%     \item \textbf{Timeline}: Weeks 1-3 cho data preparation
%     \item \textbf{Quality control}: Manual verification của ~100 samples per dataset
% \end{itemize}

% \column{0.48\textwidth}
% \textbf{Implementation Priorities}:

% \begin{enumerate}
%     \item \textbf{Week 1-2}: Download và verify Tier 1 datasets
%     \item \textbf{Week 3}: Preprocessing và standardization
%     \item \textbf{Week 4-5}: Baseline evaluation measuring consistency gap
%     \item \textbf{Week 6-7}: Pipeline implementation và full evaluation
%     \item \textbf{Week 8}: Analysis và documentation
% \end{enumerate}

% \vspace{0.3em}
% \textbf{Success Criteria}:

% \begin{itemize}
%     \item âœ" Reproducible evaluation framework
%     \item âœ" Consistency metrics alongside accuracy
%     \item âœ" Ablation study (w/ vs w/o grounding)
%     \item âœ" Error taxonomy và failure analysis
%     \item âœ" Statistical significance testing
% \end{itemize}
% \end{columns}

\vspace{0.3em}
% \begin{exampleblock}{Research Impact}
% Framework này enables systematic evaluation revealing whether visual grounding thực sự improves reasoning faithfulness hay chỉ superficially boosts accuracy metrics \parencite{wu2025gcot}.
% \end{exampleblock}
\end{frame}

%====================================================
\begin{frame}{}
\begin{center}
    \huge{TRÂN TRỌNG CẢM ƠN\\ \small TP.HCM, Tháng 10/2025}
\end{center}
\end{frame}

%====================================================
\section*{Tài liệu tham khảo}
\begin{frame}[allowframebreaks]
	\frametitle{Tài liệu tham khảo}
\printbibliography[heading=none]
\end{frame}

\end{document}